\documentclass[12pt]{article}
\usepackage{amsmath}
\usepackage{amssymb}
\usepackage{geometry}
\geometry{a4paper, margin=1in}
\usepackage{array} % Good practice to include when using >{...}
\newcommand{\pagebar}{
  \noindent\rule{\textwidth}{1pt}\\
}
\begin{document}

\noindent 535.641 Mathematical Methods Assignment 2

\vspace{1cm}

\noindent Ben Minnick \hfill Name: \underbar{Haadi Majeed}

\vspace{1cm}

\begin{center}
\Large % Make the font size larger for the table
\begin{tabular}{|c|>{\raggedleft\arraybackslash}p{2cm}|} % Changed to right-aligned p{2cm}
\hline
\rule{0pt}{1.2em} 1 & /20 \\ % Added vertical spacing with \rule
\hline
\rule{0pt}{1.2em} 2 & /20 \\
\hline
\rule{0pt}{1.2em} 3 & /20 \\
\hline
\rule{0pt}{1.2em} 4 & /20 \\
\hline
\rule{0pt}{1.2em} TOTAL & /80 \\
\hline
\end{tabular}
\end{center}

\newpage

1. In the following problems check to see if the set $S$ is a subspace of the corresponding
vector space. If it is not, explain why not. If it is, then find a basis and the dimension.
(a)
$$
S = \left\{ \begin{bmatrix} x_1 \\ x_2 \\ x_3 \end{bmatrix} \, , \ 2x_1 + x_2 = 1 \right\} \subset \mathbb{R}^3
$$
\\
Testing Relevant Axioms:\\
A4:
\begin{center}
  $\begin{bmatrix}
    0\\0\\0
  \end{bmatrix}$ (the zero vector) must be in the set\\
  $2(0) + 0 = 0 \therefore 0 = 1$ which is a contradiction, and thus not in $S$ 
\end{center}
\pagebar
(b)
$$
S = \left\{ \begin{bmatrix} x_1 \\ x_2 \end{bmatrix} \, , \ x_1 \leq x_2 \right\} \subset \mathbb{R}^2
$$
\pagebar
(c)
$$
S = \left\{ \begin{bmatrix} a & 0 \\ 0 & -a \end{bmatrix} \, , \ \forall a \in \mathbb{R} \right\} \subset \mathbb{R}^{2 \times 2}
$$
\pagebar
(d)
$$
S = \left\{ f(x) \, , \ \frac{df}{dx} = A\sin(2x), \forall A \in \mathbb{R} \right\} \subset \mathbb{R}
$$

\newpage

2. Use Cramer's rule to compute the solutions of the system
$$
\begin{aligned}
2x_1 + x_2 &= 7 \\
-3x_1 + x_3 &= -8 \\
x_2 + 2x_3 &= -3
\end{aligned}
$$

\newpage

3. Read section 1.15 of the instructor's notes. You are tasked with: 1. developing a model
for an elastic beam supported on the edges, and 2. applying this model for a specific
application, i.e. find the loading given a displacement. You decide to discretize the
beam so that there are three nodes ($P_1, P_2, P_3$). Assume Hooke's law applies.\\
(a) You have tested the beam under unit forces applied to each node and respectfully
found the following displacements:
$$
\mathbf{y}_1 = \begin{bmatrix} 2 \\ 2 \\ 0 \end{bmatrix} \, , \ \mathbf{y}_2 = \begin{bmatrix} 2 \\ 1 \\ 1 \end{bmatrix} \, , \ \mathbf{y}_3 = \begin{bmatrix} 0 \\ 1 \\ 0 \end{bmatrix}
$$
where $\mathbf{y}_1$ is the displacement that results from loading $\mathbf{F}_1 = [1, 0, 0]^T$, $\mathbf{y}_2$ is
the displacement from loading $\mathbf{F}_2 = [0, 1, 0]^T$, and $\mathbf{y}_3$ is the displacement from
loading $\mathbf{F}_3 = [0, 0, 1]^T$.
Given this information, build the model and construct the flexibility matrix.\\
\pagebar
(b) Using the model you have just derived, determine the loading that produces the
displacement
$$
\mathbf{y} = \begin{bmatrix} 4 \\ 5 \\ 3 \end{bmatrix}
$$

\newpage

4. Consider the $(x, y)$ vertices of an equilateral triangle in $\mathbb{R}^2$ given by,
$$
(0,0), (1,0), (.5, \sqrt{3}/2)
$$
Perform the following actions in order using matrix transformations:\\
(a) Reflect the triangle over the $x$-axis,\\
\pagebar
(b) Compress the triangle along the $y$-axis by a factor of 2,\\
\pagebar
(c) Rotate the triangle counterclockwise by $\pi/6$ radians about the origin, and\\
\pagebar
(d) Shear the triangle along the $x$-axis by a factor of 0.5.\\
Clearly show the matrix transformations used and compute the final set of vertices of
the triangle.

\end{document}